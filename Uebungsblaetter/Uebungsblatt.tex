\documentclass[12pt,a4paper]{article}
\usepackage[utf8]{inputenc}
\usepackage[english]{babel}
\usepackage{amsmath}
\usepackage{amsfonts}
\usepackage{amssymb}
\usepackage{graphicx}
\usepackage{mathtools} 
\usepackage{mathrsfs} 
\usepackage{dsfont}
\usepackage{enumitem}
\usepackage[left=2cm,right=2cm,top=2cm,bottom=2cm]{geometry}

\newcommand*\diff{\mathop{}\!\mathrm{d}}
\newcommand*\Diff[1]{\mathop{}\!\mathrm{d^#1}}
\newcommand{\E}{\mbox{I\negthinspace E}}
\DeclareMathOperator{\Var}{Var}
\DeclareMathOperator{\Cov}{Cov}

\title{Übungen zur Wahrscheinlichkeitstheorie}
\author{Alexander Ritz}
\date{\today}

\begin{document}
\maketitle

\begin{enumerate}

\item Aufgabe 
\begin{enumerate}[label=(\roman*)]

\item Geben Sie die kleinste und größte $\sigma$-Algebra in einer Menge $\Omega$ an.

\item Bestimmen Sie alle $\sigma$-Algebren in der vierelementigen Menge $\Omega_4:=\{a, b, c, d\}$.

\item\textbf{Sikzzieren} Sie eine $\sigma$-Algebra in der Menge $\Omega_5:= \left[0, 5 \right]$.

\item Was fällt auf beim Vergleich der $\sigma$-Algebren aus (ii) und (iii)?

\end{enumerate}


\item Aufgabe \\
Seien $\mathscr{A}$ und $\mathscr{B}$ zwei $\sigma$-Algebren in $\Omega$ mit $\mathscr{A} \in \mathscr{B}$.
\begin{enumerate}[label=(\roman*)]

\item Zeigen Sie, dass aus der $\mathscr{A}$-Messbarkeit von $f:\Omega \to \mathbb{R}$ die $\mathscr{B}$-Messbarkeit folgt.

\item Zeigen Sie, dass eine Indikatorfunktion $\mathds{1}_M:\Omega \to \mathbb{R}$ genau dann $\mathscr{A}$messbar ist, wenn $M \in \mathscr{A}$ gilt.

\item Seien $f, g: \Omega\to \mathbb{R}$ $\mathscr{A}$-messbare Funktionen. Zeigen Sie, dass die Mengen \[\{f < g\}, \{f \leq g\}, \{f = g\}\text{ und }\{ f \neq g\}\] in $\mathscr{A}$ liegen.\footnote{Dabei gelte die in der Statistik gängige Notation: $\{f < g\} = \{\omega \in \Omega \mid f(\omega) < g(\omega)\}$}


\end{enumerate}

\item Aufgabe \\
Seien $f, g: \Omega\to \mathbb{R}$ $\mathscr{A}$-messbare Funktionen.  Beweisen Sie:
\begin{enumerate}[label=(\roman*)]

\item $\alpha + \beta \cdot g$ ist $\mathscr{A}$-messbar ($\alpha, \beta \in \mathbb{R}$).

\item $f + g$ ist $\mathscr{A}$-messbar.

\item $f ^2$ ist $\mathscr{A}$-messbar.

\item $f \cdot g$ ist $\mathscr{A}$-messbar.

\end{enumerate}


\item Aufgabe \\
Beschreiben Sie die \glqq Elemente\grqq(oder auch Bestandteile) einer Zufallsvariablen und eines zugeordneten Wahrscheinlichkeitsraumes. Erläutern Sie die Notwendigkeit des Konzepts der Messbarkeit explizit.\\ \textit{Hinweis}: Gehen Sie auch auf die Begriffe \glqq Ereignis\grqq und \glqq Ergebnis\grqq ein.


\end{enumerate}
\end{document}