\documentclass[12pt,a4paper]{article}
\usepackage[utf8]{inputenc}
\usepackage[english]{babel}
\usepackage{amsmath}
\usepackage{amsfonts}
\usepackage{amssymb}
\usepackage{graphicx}
\usepackage{mathtools} 
\usepackage{mathrsfs} 
\usepackage{dsfont}
\usepackage{enumitem}
\usepackage[left=2cm,right=2cm,top=2cm,bottom=2cm]{geometry}

\newcommand*\diff{\mathop{}\!\mathrm{d}}
\newcommand*\Diff[1]{\mathop{}\!\mathrm{d^#1}}
\newcommand{\E}{\mbox{I\negthinspace E}}
\DeclareMathOperator{\Var}{Var}
\DeclareMathOperator{\Cov}{Cov}

\title{Lösungen zur Wahrscheinlichkeitstheorie}
\author{Alexander Ritz}
\date{\today}

\begin{document}
\maketitle

\begin{enumerate}

\item Aufgabe 
\begin{enumerate}[label=(\roman*)]

\item $\mathscr{A}_{min} = \{\emptyset, \Omega\}$, $\mathscr{A}_{max} = \mathcal{P}(\Omega)$, insofern existent.

\item $\mathscr{A}_{1} = \{\emptyset, \Omega_4\}$ \\ $\mathscr{A}_{2} = \{\emptyset, \Omega_4, \{a\}, \{b, c, d\}\}$ \\ $\mathscr{A}_{3} = \{\emptyset, \Omega_4, \{b\}, \{a, c, d\}\}$ \\ $\mathscr{A}_{4} = \{\emptyset, \Omega_4, \{c\}, \{a, b, d\}\}$ \\ $\mathscr{A}_{5} = \{\emptyset, \Omega_4, \{d\}, \{a, b, c\}\}$ \\ $\mathscr{A}_{6} = \{\emptyset, \Omega_4, \{a, b\}, \{c, d\}\}$ \\ $\mathscr{A}_{7} = \{\emptyset, \Omega_4, \{a, c\}, \{b, d\}\}$ \\ $\mathscr{A}_{8} = \{\emptyset, \Omega_4, \{a, d\}, \{b, c\}\}$ \\ $\mathscr{A}_{9} = \{\emptyset, \Omega_4, \{a\}, \{b\} \{c, d\}, \{b, c, d\} \{a, c, d\}, \{a, b\}\}$ \\ $\mathscr{A}_{10} = \{\emptyset, \Omega_4, \{a\}, \{c\} \{b, d\}, \{b, c, d\} \{a, b, d\}, \{a, c\}\}$ \\ $\mathscr{A}_{11} = \{\emptyset, \Omega_4, \{a\}, \{d\} \{b, c\}, \{b, c, d\} \{a, b, c\}, \{a, d\}\}$ \\ $\mathscr{A}_{12} = \{\emptyset, \Omega_4, \{b\}, \{c\} \{a, d\}, \{a, c, d\} \{a, b, d\}, \{b, c\}\}$ \\ $\mathscr{A}_{13} = \{\emptyset, \Omega_4, \{b\}, \{d\} \{a, c\}, \{a, c, d\} \{a, b, c\}, \{b, b\}\}$ \\ $\mathscr{A}_{14} = \{\emptyset, \Omega_4, \{c\}, \{d\} \{a, b\}, \{a, b, d\} \{a, b, c\}, \{c, d\}\}$ \\ $\mathscr{A}_{15} = \mathcal{P}(\Omega_4)$ 

\item Skizzieren des Konzepts trivial.

\item Offensichtlich.

\end{enumerate}


\item Aufgabe

\begin{enumerate}[label=(\roman*)]

\item Trivial, da aus Definition direkt folgend: $f$ ist $\mathscr{A}$-messbar, daher: \[
\{f \leq \alpha\} \in \mathscr{A}
\] 
Da $\mathscr{A} \subset \mathscr{B}$, folgt damit die $\mathscr{B}$-Messbarkeitsbedingung \[
\{f \leq \alpha\} \in \mathscr{B}
\]

\item Es gilt (Begründung notwendig!): \[
\{\mathds{1}_M \leq \alpha\}=\begin{cases} \Omega\quad \text{   falls } 1\leq \alpha\\
M^{\mathrm{C}} \text{ falls } 0 \leq \alpha < 1\\
\emptyset \quad\text{ falls } \alpha < 0 \end{cases}
\]
Wegen $M^{\mathrm{C}} \in \mathscr{A} \iff M \in \mathscr{A}$ liegen alle Urbilder genau dann in $\mathscr{A}$, wenn $M \in \mathscr{A}$ gilt.

\item  Lösung ohne den Hinweis relativ lang, daher in jedem Fall Äquivalenzbedingungen ausnutzen. Nach diesen liegen die Mengen $\{ f < r\}$ und $\{ r < g\}$ und damit auch die Mengen $\{ f < r\} \cap \{ r < g\}$ in $\mathscr{A}$. Aus den Eigenschaften einer $\sigma$-Algebra folgt dann, dass auch folgende Menge in  $\mathscr{A}$ liegt: \[
\bigcup\limits_{r \in \mathbb{Q}} \{ f < r\} \cap \{ r < g\} = \{ f < g\}
\]
Mit $\{ f < g\}$ liegt auch das Komplement $\{ f \geq g\}$ in $\mathscr{A}$. Analog lässt sich zeigen, dass $\{ f \leq g\}$ in $\mathscr{A}$ liegt. Da $\{ f = g\} = \{f \leq g\} \cap \{ f \geq g\}$ liegt damit auch $\{ f = g\}$ in $\mathscr{A}$.

\end{enumerate}

\item Aufgabe 

\begin{enumerate}[label=(\roman*)]

\item Für $\beta = 0$ ist die Behauptung trivial!? Es gelte daher $\beta \neq 0$. Sei $r$ eine beliebige reelle Zahl, so gilt: \[
\{\alpha + \beta \cdot g \leq r\}=\begin{cases} \{g < \frac{r - \alpha}{\beta}\}\quad \text{ falls } \beta < 0\\
\{g > \frac{r - \alpha}{\beta}\} \text{ falls } \beta < 0\\
\{\alpha < r\} \quad\text{ falls } \beta = 0 \end{cases}
\]
Unabhängig der Fallunterscheidung liegt ein Element von $\mathscr{A}$ vor. In den beiden ersten Fällen aufgrund der Messbarkeit von $g$, im dritten Fall, da $\{\alpha < r\} = \emptyset $ oder $\{\alpha < r\} = \Omega $ gilt.

\item Setzt man im vorangehenden Beweis $\beta = -1$, so folgt, dass die Funktion $\alpha - g$ für alle $\alpha \in \mathbb{R}$ messbar ist. Aus der Messbarkeit der Funktion $f$ folgt daher nach Aufgabe 2 (iii), dass für alle $\alpha \in \mathbb{R}$ die Menge \[
\{f \leq \alpha - g\} = \{f + g \leq \alpha\}
\]
in der $\sigma$-Algebra $\mathscr{A}$ liegt. Daraus folgt die Behauptung.

\item Es gilt: \[
\{f^2 \geq \alpha\}=\begin{cases} \Omega \quad \text{ falls } \alpha \leq 0\\
\{f \geq \sqrt{\alpha}\} \cup \{f \leq - \sqrt{\alpha}\} \text{ falls } \alpha > 0 \end{cases}
\]
Da die Mengen $\{f \geq \sqrt{\alpha}\}$ und $\{f \leq - \sqrt{\alpha}\}$ wegen der Messbarkeit von $f$ in $\mathscr{A}$ liegen, liegt damit auch $\{f^2 \geq \alpha\}$ in $\mathscr{A}$.

\item Da nach Teil (i) und (ii)  $\frac{1}{2}(f + g)$ und $\frac{1}{2}(f - g)$ messbar sind, folgt die Behauptung mit Teil (iii) aus der Darstellung \[
f \cdot g = \frac{1}{4}(f + g)^2 - \frac{1}{4}(f - g)^2
\]

\end{enumerate}

\item Aufgabe 

\begin{enumerate}[label=(\roman*)]

\item Offensichtlich, quasi bereits gelöst.

\item Für $z \leq 0$ gilt $\{ Y \leq z\} = \emptyset$ und daher $F_Y (z) = 0$. \\ Für $z \geq 1$ gilt $\{ Y \leq z\} = \Omega$ und daher $F_Y (z) = 1$. \\ Für $0 < z <1$ gilt 
\begin{align*} 
F_Y (z) &= P(\{ Y \leq z\})  \\ 
&=  P(\{ 1 - \exp (-c \cdot X) \leq z\}) \\
&= P(\{ X \leq -\frac{\ln (1 - z)}{c}\}) \\
&= 1- \exp (\lambda \cdot \frac{\ln (1 - z)}{c}) \\
&= 1 - (1 - z)^{\frac{\lambda}{c}}
\end{align*}
Im Spezialfall $c = \lambda$ ergibt sich also eine Gleichverteilung.

\item Partielle Integration liefert ohne Umwege $\E (X) =  \frac{1}{\lambda}$. \\ Analog liefert zweimalige partielle Integration $\Var (X) = \frac{1}{\lambda^2}$. Werte exisitieren offensichtlich nur für $\lambda > 0$.

\item Gedächtnislosigkeit wird durch die Eigenschaft beschrieben.
\begin{align*} 
P(X \leq s+ t \mid X > s) &= \frac{P(\{ X \leq s + t\} \cap \{ X > s\} )}{P(\{ X >s\})}  \\ 
&=  \frac{P(\{s < X \leq s + t\})}{1 - F_X (s)} \\
&= \frac{F_X (s + t) - F_X (s)}{1 - F_X (s)} \\
&= 1 - \exp (- \lambda \cdot t) \\
&= F_X (t) = P(X \leq t)
\end{align*}

\end{enumerate}


\item Aufgabe  \\
Analog zu besprochener Theorie. Notwendige Bestandteile, die erläutert werden sollten: Zustandsraum, Ereignis vs. Ergebnis, $\sigma$-Algebra, Wahrscheinlichkeitsmaß, Definitions- und Wertebereich der Zufallsvariable, Messbarkeit.


\end{enumerate}
\end{document}