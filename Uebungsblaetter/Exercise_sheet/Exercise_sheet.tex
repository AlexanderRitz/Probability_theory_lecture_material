\documentclass[12pt,a4paper]{article}
\usepackage[utf8]{inputenc}
\usepackage[english]{babel}
\usepackage{amsmath}
\usepackage{amsfonts}
\usepackage{amssymb}
\usepackage{graphicx}
\usepackage{mathtools} 
\usepackage{mathrsfs} 
\usepackage{dsfont}
\usepackage{enumitem}
\usepackage[left=2cm,right=2cm,top=2cm,bottom=2cm]{geometry}

\newcommand*\diff{\mathop{}\!\mathrm{d}}
\newcommand*\Diff[1]{\mathop{}\!\mathrm{d^#1}}
\newcommand{\E}{\mbox{I\negthinspace E}}
\DeclareMathOperator{\Var}{Var}
\DeclareMathOperator{\Cov}{Cov}

\title{Probability theory exercises}
\author{Alexander Ritz}
\date{\today}

\begin{document}
\maketitle

\begin{enumerate}

\item Exercise 
\begin{enumerate}[label=(\roman*)]

\item Construct the smallest and largest possible $\sigma$-algebra in a given set $\Omega$.

\item Construct all possible $\sigma$-algebras in a set of magnitude $4$: $\Omega_4:=\{a, b, c, d\}$.

\item\textbf{Sketch} what a $\sigma$-algebra in the set $\Omega_5:= \left[0, 5 \right] (\Omega_5 \in \mathbb{R})$ would conceptually look like.

\item Compare the $\sigma$-algebras constructed or sketched in (ii) und (iii). What do you notice?

\end{enumerate}


\item Exercise \\
Let $\mathscr{A}$ und $\mathscr{B}$ be two $\sigma$-algebras in $\Omega$ with $\mathscr{A} \in \mathscr{B}$.
\begin{enumerate}[label=(\roman*)]

\item Show that $\mathscr{A}$-measurability of $f:\Omega \to \mathbb{R}$ implies $\mathscr{B}$-measurability.

\item Show that an indicator function $\mathds{1}_M:\Omega \to \mathbb{R}$ is $\mathscr{A}$-measurable exactly when $M \in \mathscr{A}$.

\item Let $f, g: \Omega\to \mathbb{R}$ be $\mathscr{A}$-measurable functions. Show that the sets \[\{f < g\}, \{f \leq g\}, \{f = g\}\text{ and }\{ f \neq g\}\] are elements of $\mathscr{A}$.\footnote{With the usual notation utilised in the field of statisitcs: $\{f < g\} = \{\omega \in \Omega \mid f(\omega) < g(\omega)\}$} \\ \textit{Hint}: The $\mathscr{A}$-measurability of a function $f$ is equivalent to:\begin{align*}
\forall \alpha \in \mathbb{R} : \{ f \leq \alpha\} \in \mathscr{A} &\iff \forall \alpha \in \mathbb{R} : \{ f \geq \alpha\} \in \mathscr{A}\\ \iff \forall \alpha \in \mathbb{R} : \{ f > \alpha\} \in \mathscr{A} &\iff \forall \alpha \in \mathbb{R} : \{ f < \alpha\} \in \mathscr{A}
\end{align*}


\end{enumerate}

\item Exercise \\
Given $f, g: \Omega\to \mathbb{R}$ $\mathscr{A}$-measurable functions.  Show:
\begin{enumerate}[label=(\roman*)]

\item $\alpha + \beta \cdot g$ is $\mathscr{A}$-measurable ($\alpha, \beta \in \mathbb{R}$).

\item $f + g$ is $\mathscr{A}$-measurable.

\item $f ^2$ is $\mathscr{A}$-measurable.

\item $f \cdot g$ is $\mathscr{A}$-measurable.

\end{enumerate}

\item Exercise \\
Let $X: \Omega \to \left[ 0, + \infty \right]$ be a random variable with \textit{Exponential} distribution, implying:
\[
F_{X} (z) = P(\{ X \leq z\} ) = \int_0^z \lambda \cdot \exp(- \lambda \cdot t) \diff t = 1 - \exp (- \lambda \cdot z)
\]
\begin{enumerate}[label=(\roman*)]

\item Show that $X$ posesses a density.

\item Derive the distribution function of the random variable $Y:= 1 - \exp (-c \cdot X)$. What special case is constructed by setting $c$ equal to $\lambda$?

\item Calculate, insofar either exists, expected value and variance of $X$.

\item Show that, for all $s, t \geq 0$
\[
P(X \leq s + t \mid X > s) = P(X \leq t)
\]
Interpret the statement.

\end{enumerate}


\item Exercise \\
Describe the \glqq elements\grqq(or rather components) of a random variable and its associated probability space. Evaluate and explicitly describe the necessity of the often assumed measurability property.\\ \textit{Hint}: The concept of \glqq events\grqq und \glqq results\grqq should be considered as well.


\end{enumerate}
\end{document}