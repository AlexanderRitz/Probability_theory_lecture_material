\documentclass[12pt]{article} 

\usepackage[utf8]{inputenc} 


\usepackage[margin=3.3cm]{geometry} 
\geometry{a4paper} 
\usepackage[parfill]{parskip} 


\usepackage{graphicx} 
\usepackage{booktabs}
\usepackage{array} 
\usepackage{paralist} 
\usepackage{verbatim} 
\usepackage{mathtools}
\usepackage[hidelinks]{hyperref}
\usepackage{german}

\urlstyle{same}
\usepackage{float}
\usepackage{caption}
\usepackage{subcaption}
\usepackage{booktabs}
\usepackage{setspace}
\usepackage{eurosym}
\usepackage{listings} 
\usepackage{amsmath}
\usepackage{amssymb}
\usepackage{tablefootnote}

\usepackage{mathtools} 
\usepackage{mathrsfs} 

\usepackage{fancyhdr} 
\pagestyle{fancy} 
\renewcommand{\headrulewidth}{0pt} 
\lhead{}\chead{}\rhead{}
\lfoot{}\cfoot{\thepage}\rfoot{}

\usepackage{sectsty}
\allsectionsfont{\sffamily\mdseries\upshape}


\usepackage[nottoc,notlof,notlot]{tocbibind} 
\usepackage[titles]{tocloft} 
\renewcommand{\cftsecfont}{\rmfamily\mdseries\upshape}
\renewcommand{\cftsecpagefont}{\rmfamily\mdseries\upshape} 
\usepackage{setspace}
\onehalfspacing

\newtheorem{definition}{Definition}
\begin{document}

\pagenumbering{arabic}

\section{Zustandsräume}

$\Omega$ bezeichne im Folgenden eine Menge von möglichen Umweltzuständen $\omega$, deren Eintritt
nicht vorhersehbar ist. Ein Umweltzustand $\omega \in \Omega$
 ist dabei zu verstehen als Zusammenfassung
aller Zustände und Konstellationen, welche die betrachteten Größen beeinflussen. Die
Menge $\Omega$ wird als Zustands- oder Ergebnisraum, Teilmengen von $\Omega$ 
 werden als Ereignisse bezeichnet. Ist der Zustand $\omega$ eingetreten, so sagen wir, \glqq Das Ereignis A ist eingetreten\grqq ,
wenn $\omega \in A$ gilt. Im Fall $\omega \notin A$ sagt man, \glqq Das Ereignis A ist nicht eingetreten\grqq . Ein Ereignis
wird als bekannt bezeichnet, wenn es eingetreten oder nicht eingetreten ist.

\section{$\sigma$-Algebren}
Mit dem Eintreten eines Zustandes $\omega$ sind nicht nur einzelne Ereignisse sondern auch zusammengesetzte
Ereignisse bekannt \footnote{Diese informationstheoretische Interpretation von $\sigma$-Algebren geht auf den deutschen Mathematiker Klaus Schindler zurück.}. Sind nämlich $A$ und $B$ bekannte Ereignisse, so gilt dies aus mengentheoretischen Gründen z.B. auch für $A^{\mathrm{C}}$, $A \cap B$ oder $A \cup B$. Ein System $\mathscr{A}$ von
beobachtbaren Ereignissen, das diese mehr oder minder naheliegenden mengentheoretischen
Eigenschaften besitzt, wird als $\sigma$-Algebra bezeichnet. Genauer definiert man: \\

Ein System $\mathscr{A}$ von Teilmengen der Menge $\Omega$ heißt $\sigma$-Algebra in $\Omega$, wenn es folgende
Eigenschaften erfüllt:
\begin{itemize}
\item{$\Omega \in \mathscr{A}$}
\item{$A \in \mathscr{A} \implies A^{\mathrm{C}} \in \mathscr{A}$}
\item{$A_1, A_2, A_3, ... \in \mathscr{A} \implies \bigcup\limits_{l \in \mathbb{N}} A_l \in \mathscr{A}$}
\end{itemize}
Ein Paar $(\Omega, \mathscr{A})$, bestehend aus einem Zustandsraum $\Omega$
 und einer $\sigma$-Algebra $A \subset \mathscr{P}(\Omega)$ wird
als Messraum bezeichnet.\footnote{Man beachte, dass diese \glqq Vereinigungsstabilität\grqq (die letzte Eigenschaft) nur für abzählbare Vereinigungen gefordert wird. Überabzählbare Vereinigungen von Elementen der $\sigma$-Algebra $\mathscr{A}$ liegen i.A. nicht mehr in $\mathscr{A}$! Entsprechende Vorsicht ist ander Schnittstelle zwischen diskreten und stetigen Zufallsvariablen geboten.}

\section{Wahrscheinlichkeitsräume}
Zwar kann man i.A. nicht voraussagen, welche Ereignisse zukünftig eintreten, jedoch ist
es oft möglich, eine Einschätzung abzugeben, mit welchen Ereignissen in einer gegebenen
$\sigma$-Algebra eher zu rechnen ist und welche weniger plausibel sind. Dies wird präzisiert durch
die Angabe von Werten zwischen $0$ und $1$, die man als Wahrscheinlichkeit bezeichnet. Ist
$A \subset \Omega$ ein Ereignis, so bezeichnet $P(A)$ die Wahrscheinlichkeit dafür, dass das Ereignis $A$ eintritt. Man nennt $P$ ein
Wahrscheinlichkeitsmaß. Aufgrund mengentheoretischer\footnote{Man könnte auch von \textit{maß}theoretischen Überlegungen sprechen.} Überlegungen ist es sinnvoll, von
diesem Maß gewisse Eigenschaften zu fordern. Sind z.B. $A$ und $B$ disjunkte Ereignisse, sollte
$P(A \cup B) = P(A)+P(B)$ gelten. Außerdem sollte die Wahrscheinlichkeit für alle Ereignisse
aus der gegebenen $\sigma$-Algebra berechnet werden können. Dies führt zu folgender Definition.\\

Sei $A$ eine $\sigma$-Algebra im Zustandsraum $\Omega$. Eine Funktion $P : \mathscr{A} \to [0, 1]$ heißt Wahrscheinlichkeitsmaß
auf $\mathscr{A}$, wenn gilt
\begin{itemize}
\item{$P(\Omega) = 1$}
\item{$P$ ist $\sigma$-additiv, d.h. für jede Folge paarweise disjunkter Mengen $A_1, A_2, . . .$ gilt: \[
P\left( \dot{\bigcup\limits_{i \in \mathbb{N}}}A_i \right) = \sum_{i \in \mathbb{N}}P(A_i)
\]}
\end{itemize}
Das Tripel $(\Omega, \mathscr{A}, P)$ wird als Wahrscheinlichkeitsraum bezeichnet.

\section{Zufallsvariablen und Messbarkeit}
So elegant und allgemein das Konzept des Wahrscheinlichkeitsraumes gehalten ist\footnote{Dieses grundlegende axiomatische Modell geht auf den russischen Mathematiker Kolmogoroff zurück.}, so wenig
praktikabel erscheint es, da eine vollständige Bestimmung des gesamten Zustandsraumes $\Omega$
auf Grund seiner Komplexität i.A. unmöglich oder viel zu aufwändig wäre. Man wird sich
daher nur auf die Daten bzw. Ereignisse konzentrieren, an denen man wirklich interessiert ist.
Diese Größen, wie z.B. Aktienkurse oder Temperaturen, deren Werte direkt vom jeweiligen
zufälligen zukünftigen Umweltzustand abhängen, bezeichnet man als Zufallsgrößen.

Eine Abbildung auf dem Zustandsraum $\Omega$ \[
Z : \Omega \to \mathbb{R}^d \quad \text{  mit  }\quad \omega \mapsto Z(\omega)
\]
bezeichnet man als Zufallsgröße. Im Fall $d=1$ spricht man von einer Zufallsvariable. Im Fall
$d>1$ ist $Z$ ein Vektor von Zufallsvariablen, d.h. es gilt $Z = (Z_1, . . . , Z_d)$ und man spricht
von einem $d$-dimensionalen Zufallsvektor.\footnote{Man beachte, dass sich in dieser Definition auf reellwertige Zufallsgrößen beschränkt wurde, jedoch lässt sich das Konzept auf triviale Weise verallgemeinern.}
\\
Statt alle möglichen Ereignisse zu betrachten, wird man seine Aufmerksamkeit auf die
Ereignisse konzentrieren, die mit einer gegebenen Zufallsgröße $Z$ zu tun haben. Da auf
Grund des vorher schon erwähnten nicht vorhersehbaren stochastischen Charakters unserer
Umwelt nur eine Bandbreite von in Frage kommenden zukünftigen Umweltzuständen angegeben
werden kann (Ereignisse), ist es bei einer gegebenen ZV $Z$ auch sinnvoller, nach dem
Eintreten eines Intervalls von Werten von $Z$, statt nach dem Eintreten einzelner Werte zu
fragen. Von Interesse sind also vor allem die Ereignisse in $\Omega$, für die $Z$ Werte innerhalb eines
vorgegebenen Intervalls annimmt, also die Urbilder \[
Z^{-1}(\left] -\infty, x \right] ) = \{ \omega \in \Omega \mid -\infty < Z(\omega) \leq x \} =: \{ Z \leq x \}
\]
\glqq Beherrschbar\grqq  ist eine Zufallsgröße $Z$ nur, wenn diese Ereignisse beobachtbar bzw. \glqq messbar\grqq
sind, d.h. wenn man die Eintrittswahrscheinlichkeit dieser Ereignisse berechnen kann.
Mathematisch bedeutet dies, dass sie im Definitionsbereich des Wahrscheinlichkeitsmaßes
liegen, also Elemente der $\sigma$-Algebra sein müssen. Diese Messbarkeit ist eine Minimalforderung,
die wir in Zukunft von allen Zufallsgrößen verlangen werden.

Eine Zufallsgröße $Z : \Omega \to \mathbb{R}^d $ heißt messbar bzgl. der $\sigma$-Algebra $\mathscr{A}$, wenn gilt:
\[
\forall x \in \mathbb{R}^d : \{Z \leq x\} \in \mathscr{A}
\]
Hierbei ist $\{Z \leq x\} \in \mathscr{A}$ eine Kurznotation für die Menge der Umweltzustände $\omega \in \Omega$, die bei der
Funktion $Z = (Z_1, . . . , Z_d)$ zu Werten unterhalb von $x = (x_1, . . . , x_d)$ führen, d.h.:
\[
\{Z \leq x\} = \{\omega \in \Omega \mid Z(\omega) \leq x\} = \{\omega \in \Omega \mid Z_1(\omega)\leq x_1, ..., Z_d(\omega)\leq x_d\}
\]

\end{document}